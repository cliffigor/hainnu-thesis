%------------------------------------------------------------
%    MathDef.tex   你可以修改或设定自己的数学公式环境定义
%------------------------------------------------------------

\usepackage[electronic]{ifsym}
\usepackage{indentfirst}
\usepackage{latexsym}

%%\usepackage{unicode-math}
%\usepackage{mathpazo}
%%\usepackage[mathpazo]{flexisym}
\usepackage{bm}
\usepackage{bbm}
\usepackage{mathrsfs}

\usepackage{amsmath}
%\usepackage{amssymb}
\usepackage{amsfonts}
%\usepackage{amsthm}
\usepackage{amscd}
\usepackage{pifont}
%\usepackage{wasysym}
\usepackage[mathscr]{euscript}

\usepackage[all]{xy}  % 画交换图


\newcommand{\op}[1]{\mathcal{#1}}
\newcommand{\Real}{\mathbb{R}}
\newcommand{\Complex}{\mathbb{C}}
\newcommand{\Proj}{\mathbb{P}}
\newcommand{\field}[1]{\mathbb{#1}}
\newcommand{\cc}[1]{\overline{#1}}   % complex congugate
\newcommand{\hc}[1]{{#1}^{\dagger}}  % Hermitian congugate

%%%%%%%%%%%%%%%%%%%%%%%%%%%%%%%%%
%%   Operators
%%%%%%%%%%%%%%%%%%%%%%%%%%%%%%%%%
\DeclareMathOperator{\Ker}{\mathrm{Ker}}
\DeclareMathOperator{\Image}{\mathrm{Im}}
\DeclareMathOperator{\argmin}{\mathrm{argmin}}
\DeclareMathOperator{\SSE}{\mathrm{SSE}}
\DeclareMathOperator{\dist}{\mathrm{dist}}
\DeclareMathOperator*{\union}{\cup}
\DeclareMathOperator{\polydeg}{\mathrm{deg}}
\DeclareMathOperator{\lspan}{\mathrm{span}}
\DeclareMathOperator{\diag}{\mathrm{diag}}
\DeclareMathOperator{\rank}{\mathrm{rank}}
\DeclareMathOperator{\trace}{\mathrm{tr}}
\DeclareMathOperator{\Dim}{\mathrm{dim}}
\DeclareMathOperator{\Dom}{\mathrm{Dom}}
\DeclareMathOperator{\Range}{\mathrm{Range}}
\DeclareMathOperator{\Null}{\mathrm{Null}}
\DeclareMathOperator{\FFT}{\mathrm{FFT}}
\DeclareMathOperator{\DFT}{\mathrm{DFT}}    % Discrete Fourier Transformation
\DeclareMathOperator{\id}{\mathbbm{1}}
\DeclareMathOperator{\dif}{\mathrm{d}}
\DeclareMathOperator{\Hom}{\mathrm{Hom}}
\DeclareMathOperator{\sign}{\mathrm{sign}}
\DeclareMathOperator{\Ann}{\mathrm{Ann}}
\DeclareMathOperator{\dB}{\mathrm{dB}}
\DeclareMathOperator{\GF}{\mathbb{GF}} % Galois Filed
\DeclareMathOperator{\sinc}{\mathrm{sinc}}      % sin count function, sinc(x) = sin (\pi x)/(\pi x)
\DeclareMathOperator{\rect}{\mathrm{rect}}      % regutangular impulse
\DeclareMathOperator{\erf}{\mathrm{erf}}
\DeclareMathOperator{\erfc}{\mathrm{erfc}}
\DeclareMathOperator{\Q}{\mathrm{Q}}
\DeclareMathOperator{\const}{\mathrm{const}}

\DeclareMathOperator{\vecs}{\mathrm{vecs}}
\DeclareMathOperator{\Zern}{\mathrm{Z}}
\DeclareMathOperator{\grad}{\overrightarrow{\mathrm{grad}}}
\DeclareMathOperator{\dive}{\mathrm{div}}
\DeclareMathOperator{\curl}{\overrightarrow{\mathrm{rot}}}

\newcommand{\atan}{\mathrm{atan2}}

\newcommand{\mapstset}[2]{\left.{#1}\right|_{#2}}


%%%%%%%%%%%%%%%%%%%%%%%%%%%%%%%%%%%%%%%%%%%%%%%%%%%%%%%%%%%%%%%
%  CV notations
%%%%%%%%%%%%%%%%%%%%%%%%%%%%%%%%%%%%%%%%%%%%%%%%%%%%%%%%%%%%%%%
\newcommand{\univset}[1]{\mathscr{#1}} % universal set
\newcommand{\setclass}[1]{\mathcal{#1}} % set class
\newcommand{\manifold}[1]{\mathcal{#1}}
\newcommand{\chart}[1]{\mathscr{#1}}  % coordinate chart in manifold
\newcommand{\manpt}[1]{\mathrm{#1}}     % point in a manifold
\newcommand{\vecfield}{\mathscr{X}}
\newcommand{\TpM}[2]{\mathscr{T}_{#1}(#2)}
\newcommand{\cTpM}[2]{\mathscr{T}^*_{#1}(#2)}
\newcommand{\tanspace}[1]{\mathscr{T}(#1)}          % tangent space
\newcommand{\cotspace}[1]{\mathscr{T}^*(#1)}       % cotangent space
\newcommand{\tanbundle}[1]{\mpair{#1}{\mathscr{T}(#1)}} % tangent fibre bundle
\newcommand{\ctanbundle}[1]{\mpair{#1}{\mathscr{T}^*(#1)}} % cotangent fibre bundle
\newcommand{\Diff}{\mathrm{Diff}} % diffeomorphism

\newcommand{\img}[1]{\mathcal{#1}}      % image frame variable
\newcommand{\wopt}[1]{\bf{#1}}          % world point  4-vector
\newcommand{\impt}[1]{\bf{#1}}          % image广。矩阵求导与微分,是计算机视觉、信号处理以及其他很多工程领域都有着广泛的应用。 point  3-vector
\newcommand{\imli}[1]{\bf{#1}}          % image line   3-vector
\newcommand{\cam}[1]{\bf{#1}}       % camera projection 3-by-4 matrix
\newcommand{\IAC}[1]{\tinv{#1}\inv{#1}}
\newcommand{\conic}[1]{\mathcal{#1}}    % conics
\DeclareMathOperator{\MOSAIC}{MOSAIC}  % 全景拼接
\DeclareMathOperator{\GetStrip}{GetStrip} % 图像条带选取
\newcommand{\strip}[1]{\overline{#1}} % image strip
\DeclareMathOperator*{\mosaic}{\sqcup}
\newcommand{\woptl}[1]{\scrut{\mathbf{#1}}{loc}}   %local frame, world coordinate point  4-vector
\newcommand{\woptg}[1]{\scrut{\mathbf{#1}}{glb}}   %global frame, world coordinate point  4-vector
\newcommand{\woptsl}[1]{\scrut{\mathfrak{#1}}{loc}}   %local frame, world coordinate point  4-vector
\newcommand{\woptsg}[1]{\scrut{\mathfrak{#1}}{glb}}   %global frame, world coordinate point  4-vector
\newcommand{\imsz}[2]{\texttt{#1} \times \texttt{#2}}  % image size




\DeclareMathOperator{\Int}{\mathrm{int}} % interior of a set
\DeclareMathOperator{\Ext}{\mathrm{ext}} % exterior of a set
\DeclareMathOperator{\bry}{\partial} % boundary operator

\newcommand{\gso}{\mathrm{g^2o}}

 
\DeclareMathOperator{\mtv}{\mathrm{vec}}    % 矩阵按列堆栈生成向量
\DeclareMathOperator{\smtv}{\mathrm{vecs}}  % 对称矩阵按行生成向量
\newcommand{\vtm}[1]{\inv{\mtv}(#1)}        % 向量整形生成矩阵
\newcommand{\vtsm}[1]{\inv{\smtv}(#1)}      % 向量按行生成对称矩阵
\newcommand{\oplift}[1]{{#1}^\wedge}   % 三维向量提升为三阶反对称矩阵
\newcommand{\opdown}[1]{{#1}^\lor}     % 三阶反对称矩阵降格为三维向量 
\newcommand{\vtam}[1]{\scrdm{\left[#1\right]}{\times}}      % 三维向量整形为三阶反对称矩阵

\newcommand{\liebraket}[2]{\left[#1, #2\right]} % Lie括号

%\DeclareMathOperator{\vecm}{vec}                     % 按列向量堆栈(一般矩阵)
%\DeclareMathOperator{\unvm}{unvec}
%\DeclareMathOperator{\vecs}{vecs} 
%\newcommand{\vecmat}[1]{\vecm{\left( {#1} \right)}}
%\newcommand{\vecsmat}[1]{\vecs{\left( {#1} \right)}}
%\newcommand{\vecasym}[1]{[#1]_\times}   % antisymmetric matrix from a vector
%\newcommand{\matTOvec}[1]{\vecm{\left( #1 \right)}}
%\newcommand{\smatTOvec}[1]{\vecs{\left( #1 \right)}}
%\newcommand{\vecTOasym}[1]{[#1]_{_\times}}

%\newcommand{\matTOvec}[1]{\vecm{\left( #1 \right)}}
\newcommand{\smatTOvec}[1]{\vecs{\left( #1 \right)}}
\newcommand{\vecTOasym}[1]{[#1]_{_\times}}


\newcommand{\scrdm}[2]{{#1}_{#2}}
\newcommand{\scrum}[2]{{#1}^{#2}}
\newcommand{\scrudm}[3]{{#1}^{#2}_{#3}}

\newcommand{\scrdt}[2]{{#1}_{\mathrm{#2}}}
\newcommand{\scrut}[2]{{#1}^{\mathrm{#2}}}
\newcommand{\scrudt}[3]{{#1}^{\mathrm{#2}}_\mathrm{#3}}


\renewcommand{\bf}[1]{\mathbf{#1}}
\newcommand{\hompt}[1]{\bf{#1}}      % homegenous coordinate of a point
\renewcommand{\vec}[1]{\bm{#1}}      % \bm
\newcommand{\mat}[1]{\bm{#1}}        % \bm
\newcommand{\hmat}[1]{\mathbf{#1}}   % homegeneous transform


\newcommand{\mean}[1]{\overline{#1}}
\newcommand{\card}[1]{\left| #1 \right|}
\newcommand{\mfloor}[1]{ \left\lfloor {#1} \right\rfloor }
\newcommand{\mpair}[2]{\left({#1}, {#2} \right)}
\newcommand{\mtrip}[3]{\left({#1}; {#2}, {#3}\right)}
\newcommand{\pairmap}[2]{\left\langle {#1}, {#2}\right\rangle}
\newcommand{\triplet}[3]{\left({#1}, {#2}, {#3}\right)}
\newcommand{\manchart}[3]{\left\langle {#1}, {#2(#1)}, {#3}\right\rangle}
\newcommand{\tranmap}[3]{\scrudm{#1}{#2}{#3}}
\newcommand{\seq}[1]{\left\langle {#1} \right\rangle}

\newcommand{\eps}{\varepsilon}
\newcommand{\To}{\longrightarrow}
\newcommand{\mi}{\mathrm{i}}
\newcommand{\mj}{\mathrm{j}}
\newcommand{\mk}{\mathrm{k}}
\newcommand{\me}{\mathrm{e}}

\newcommand{\fracode}[2]{\frac{\dif {#1}}{\dif {#2}}}         % ordinary differential operator
\newcommand{\pdif}[1]{\mathrm{D}_{#1}}
\newcommand{\fracpde}[2]{\frac{\partial {#1}}{\partial {#2}}} % partial differential operator
\newcommand{\fracpderow}[2]{\partial {#1}/\partial {#2}}
\newcommand{\fracoderow}[2]{\dif {#1}/\dif {#2}}
\newcommand{\fracpdemix}[3]{\frac{\partial^2 {#1}}{\partial {#2} \partial {#3}}}
\newcommand{\lap}[2]{\frac{\partial^2 {#1}}{\partial {#2}^2}}
\newcommand{\laprow}[2]{\partial^2 {#1}/\partial {#2}^2}
\newcommand{\secode}[2]{\frac{\dif^2 {#1}}{\dif {#2}^2}}
\newcommand{\set}[1]{\left\{ #1 \right\}}
\newcommand{\setind}[1]{\bf{1}_{#1}}   % set index function
\newcommand{\abs}[1]{\left| #1 \right|}
\newcommand{\absvec}[1]{\left| \bf{#1} \right|}
\newcommand{\ket}[1]{|#1 \rangle}
\newcommand{\bra}[1]{\langle #1 |}
\newcommand{\braket}[2]{ \langle #1 |  #2 \rangle}
\newcommand{\norm}[1]{\left\lVert #1 \right\rVert}
\newcommand{\normF}[1]{\norm{#1}_{\small{F}}}
\newcommand{\trsp}[1]{{#1}^{\small{\textsf{T}}}}
\newcommand{\itrsp}[1]{{#1}^{-\small{\textsf{T}}}}
\newcommand{\htrsp}[1]{{#1}^{\small{\textsf{H}}}}
\newcommand{\inv}[1]{#1^{-1}}
\newcommand{\ginv}[1]{#1^+}    % Moore-Penrose (general) inverse
\newcommand{\tinv}[1]{{#1}^{-\small{\textsf{T}}}}

\newcommand{\lhom}[2]{\mathscr{L}(#1;#2)} % 线性同态
\newcommand{\Grgl}[1]{\mathrm{gl}(#1)}
\newcommand{\GrGL}[2]{\mathrm{GL}(#1,\mathbb{#2})}
\newcommand{\GrPGL}[2]{\mathrm{PGL}(#1,\mathbb{#2})}
\newcommand{\GrGT}[2]{\mathrm{GT}(#1,\mathbb{#2})}
\newcommand{\GrGA}[2]{\mathrm{GA}(#1,\mathbb{#2})}
\newcommand{\GrSL}[2]{\mathrm{SL}(#1,\mathbb{#2})}
\newcommand{\GrO}[2]{\mathrm{O}(#1,\mathbb{#2})}
\newcommand{\GrSO}[2]{\mathrm{SO}(#1,\mathbb{#2})}
\newcommand{\GrE}[1]{\mathrm{E}(#1,\mathbb{R})}
\newcommand{\GrSE}[1]{\mathrm{SE}(#1,\mathbb{R})}
\newcommand{\GrGS}[1]{\mathrm{GS}(#1,\mathbb{R})}
\newcommand{\GrSim}[1]{\mathrm{Sim}(#1,\mathbb{R})}

\newcommand{\GrU}[1]{\mathrm{U}(#1)}
\newcommand{\GrSU}[1]{\mathrm{SU}(#1)}
\newcommand{\GrSp}[2]{\mathrm{Sp}(#1,\mathbb{#2})}

\DeclareMathOperator{\Ad}{\mathrm{Ad}}      % Lie群元素的伴随表示, Ad(g)
\DeclareMathOperator{\ad}{\mathrm{ad}}      % Lie群元素的伴随表示, ad(X)
\newcommand{\lalg}[1]{\mathrm{#1}} % Lie代数 g
\newcommand{\lalgo}[2]{\mathrm{o}(#1,\mathbb{#2})}    % Lie代数 o(n,R)
\newcommand{\lalgse}[1]{\mathrm{se}(#1,\mathbb{R})} % Lie代数se(n,R)
\newcommand{\lalgso}[2]{\mathrm{so}(#1,\mathbb{#2})} % Lie代数so(n,R)

\newcommand{\ES}[3]{\mathbb{#1}^{{#2}\times {#3}}}     % Euclidean space
\newcommand{\PS}[1]{\Proj^{#1}}       % projective space
\newcommand{\RealPS}[1]{\mathrm{R}\PS{#1}}
\newcommand{\CompPS}[1]{\mathrm{C}\PS{#1}}
\newcommand{\lspace}[1]{\mathscr{#1}}               % linear space
\newcommand{\lsvf}[2]{\lspace{#1}_{\mathbb{#2}}}    % linear space V on field F$$
\newcommand{\dlspace}[1]{\mathscr{#1}^*}            % dual linear space
\newcommand{\dlsvf}[2]{\lspace{#1}^*_{\mathbb{#2}}}


\newcommand{\edf}[2]{\mathscr{A}^{#1}(\mathcal{#2})}
\newcommand{\efbd}[3]{\Lambda^{#1}(\mathcal{#2}_{#3}^*)}
\newcommand{\efb}[3]{\Lambda^{#1}(\mathcal{#2}_{#3})}
\DeclareMathAlphabet{\mathsfsl}{OT1}{cmss}{m}{sl}
\newcommand{\tensor}[1]{\mathsfsl{#1}}    % tensor
\newcommand{\quat}[1]{\mathbbm{#1}}       % quaternions
\newcommand{\prodku}[2]{{#1}^{^{[#2]}}}
\newcommand{\prodkd}[2]{{#1}_{_{[#2]}}}
\newcommand{\Zernpoly}[2]{\operatorname{Z}_{#1}^{#2}}
\newcommand{\Radipoly}[2]{\operatorname{R}_{#1}^{#2}}
\newcommand{\Hypergeo}[3]{F\left(\left. \frac{#1}{#2}\right| {#3} \right)}
\DeclareMathOperator{\Prob}{\mathrm{Pr}}
\DeclareMathOperator{\E}{\mathrm{E}}
\DeclareMathOperator{\Cov}{\mathrm{Cov}}
\DeclareMathOperator{\D}{\mathrm{Var}}
\newcommand{\Expt}[1]{\E\set{#1}}
\newcommand{\Var}[1]{\D\set{#1}}
\newcommand{\ERD}{\mathrm{ERD}}
\newcommand{\Eig}{\mathrm{Eig}}
\newcommand{\err}{\epsilon}  % error

\newcommand{\intset}[2]{\left\{#1,\cdots,#2\right\}} % set of integers
